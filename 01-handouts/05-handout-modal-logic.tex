\PassOptionsToPackage{table}{xcolor}
\documentclass[nobib,nofonts]{tufte-handout}

%\geometry{showframe} % display margins for debugging page layout

%%% MF additions
% \usepackage[table]{xcolor}
\usepackage[nographicx, nohyperref, nosubcaption, nogb4e, nobiblatex]{../99-auxiliary-files/00-mypackages}
\usepackage{../99-auxiliary-files/00-mycommands}
\usepackage{../99-auxiliary-files/00-myenvironments}

\usepackage{titlesec}
\usepackage{etoolbox}
\usepackage{tikz-qtree}
\usepackage{subcaption}
\usepackage{CSPstyles}

\usepackage{pgfplots}
% externalize PGF plots
% \usepgfplotslibrary{external}
% \tikzexternalize

% \titleformat{\section}
% {\large\bfshape}{\thesection}{1em}{}

\setcounter{secnumdepth}{5}
\renewcommand\thesection{\arabic{section}}

% this length controls tha hanging indent for titles
% change the value according to your needs
\newlength\titleindent
\setlength\titleindent{0.7cm}

\pretocmd{\paragraph}{\stepcounter{subsection}}{}{}
\pretocmd{\subparagraph}{\stepcounter{subsubsection}}{}{}

\titleformat{\chapter}[block]
  {\normalfont\huge\bfseries}{}{0pt}{\hspace*{-\titleindent}}

\titleformat{\section}
  {\normalfont\Large\itshape}{\llap{\parbox{\titleindent}{\thesection\hfill}}}{0em}{}

\titleformat{\subsection}
  {\normalfont\itshape}{\llap{\parbox{\titleindent}{\thesubsection\hfill}}}{0em}{}

\titleformat{\subsubsection}
  {\normalfont\normalsize\itshape}{\llap{\parbox{\titleindent}{\thesubsubsection}}}{0em}{}

\titleformat{\paragraph}[runin]
  {\normalfont\normalsize\itshape}{}{-0.7cm}{}[\xspace \ \ \ \ ]

\titleformat{\subparagraph}[runin]
  {\normalfont\normalsize}{\llap{\parbox{\titleindent}{\thesubsubsection\hfill}}}{0em}{}

\titlespacing*{\chapter}{0pt}{0pt}{20pt}
\titlespacing*{\subsubsection}{0pt}{3.25ex plus 1ex minus .2ex}{1.5ex plus .2ex}
\titlespacing*{\paragraph}{0pt}{3.25ex plus 1ex minus .2ex}{0em}
\titlespacing*{\subparagraph}{0pt}{3.25ex plus 1ex minus .2ex}{0em}

\DefineNamedColor{named}{mygray2}{cmyk}{0.55,0.25,0.25,0.25}
\newcommand{\mygray}[1]{\textcolor{mygray2}{#1}}

%%% Tufte style
\usepackage{graphicx} % allow embedded images
  \setkeys{Gin}{width=\linewidth,totalheight=\textheight,keepaspectratio}
  \graphicspath{{graphics/}} % set of paths to search for images

\usepackage{fancyvrb} % extended verbatim environments
  \fvset{fontsize=\normalsize}% default font size for fancy-verbatim environments

% Standardize command font styles and environments
\newcommand{\doccmd}[1]{\texttt{\textbackslash#1}}% command name -- adds backslash automatically
\newcommand{\docopt}[1]{\ensuremath{\langle}\textrm{\textit{#1}}\ensuremath{\rangle}}% optional command argument
\newcommand{\docarg}[1]{\textrm{\textit{#1}}}% (required) command argument
\newcommand{\docenv}[1]{\textsf{#1}}% environment name
\newcommand{\docpkg}[1]{\texttt{#1}}% package name
\newcommand{\doccls}[1]{\texttt{#1}}% document class name
\newcommand{\docclsopt}[1]{\texttt{#1}}% document class option name
\newenvironment{docspec}{\begin{quote}\noindent}{\end{quote}}% command specification environment

\newcommand{\proplog}{\acro{PropLog}}
\newcommand{\predlog}{\acro{PredLog}}
\newcommand{\EFSQ}{\ensuremath{\mathit{EFSQ}}\xspace}

%%%%%%%%%%%%%%%%%%%%%%%%%%%%%%%%%%%%%%%%%%%%%%%%%%

% \usepackage[sc,osf]{mathpazo}
% \linespread{1.05}

\usepackage{turnstile}
\usetikzlibrary{patterns}
\usetikzlibrary{calc}

% TODO sort out what's needed there;
% TODO make consistent with notation in other handouts

\newcommand{\Props}{\ensuremath{\Phi}}
\newcommand{\Lang}{\ensuremath{\mathfrak{L}}}

\newcommand{\ts}{\ensuremath{\thinspace}}

\newcommand{\mybar}{\ensuremath{\ | \ }}

\renewcommand{\phi}{\ensuremath{\varphi}}

\newcommand{\ra}{\ensuremath{\rightarrow}}

\newcommand{\Bel}{\ensuremath{\Box}}
\newcommand{\CB}{\ensuremath{CB}}
\newcommand{\Poss}{\ensuremath{\Diamond}}

\newcommand{\Model}{\ensuremath{\mathcal{M}}}
\newcommand{\Models}{\ensuremath{\mathfrak{M}_{\mathcal{P},\mathcal{A}}}}
\newcommand{\AS}{\ensuremath{\mathcal{A}}}
\newcommand{\ASs}{\ensuremath{\mathfrak{A}_{\Props,i}}}
\renewcommand{\SS}{\ensuremath{\mathcal{S}}}
\newcommand{\SSs}{\ensuremath{\mathfrak{S}_{\Props,i}}}
\newcommand{\MSS}{\ensuremath{\mathcal{S}^{l}}}
\newcommand{\MSSs}{\ensuremath{\mathfrak{S}^{l}_{\Props,i}}}


\newcommand{\Aware}{\ensuremath{\mathrm{A}}}

% \newcommand{\Coloneqq}{\ensuremath{\mathrel{\mathop:}\mathrel{\mathop:}=}}
% \newcommand{\modelsweak}{\ensuremath{\sdtstile{}{*}}}
% \newcommand{\modelsweak}{\ensuremath{\mathrel{\mathop{\models^{*}}}}}
\newcommand{\modelsweak}{\ensuremath{\mathrel{\mathop{|\hspace{-0.4em}\approx}}}}
\newcommand{\modelsthree}{\ensuremath{\sttstile{}{}}}


\newcommand{\lga}{\acro{lga}}
\newcommand{\sss}{\acro{sss}}
\newcommand{\modlog}{\acro{ModLog}}

%%%%%%%%%%%%%%%%%%%%%%%%%%%%%%%%%%%%%%%%%%%%%%%%%%

\title{Modal logic}

\author{Michael Franke}

\date{} % without \date command, current date is supplied

\begin{document}

\maketitle

\begin{abstract}
\noindent
Notions covered:
language of modal logic,
(pointed) modal models,
truth in pointed models,
belief and knowledge models.
\end{abstract}

\noindent Propositional logic and predicate logic give us formulas to talk about the \emph{here and now}.
For example, the sentence:
%
\begin{align*}
  \underbrace{\text{The earth is round}}_{p}
\end{align*}
%
is to be understood as a claim about how the world is like.
If we wanted to check whether this sentence is true or false, we would have to go search \emph{this} world, the \emph{actual world} that we live in.

Yet, much of human language involves imaginary circumstances, ways in which the world could have been but isn't.
For example, in order to evaluate whether the following sentences are true:
\begin{align*}
  \underbrace{\text{ It is logically necessary that }}_{\Box}
  \underbrace{\text{ the earth is round }}_{p}
\end{align*}
\begin{align*}
  \underbrace{\text{ Alex believes that }}_{\Box_{a}}
  \underbrace{\text{ the earth is round }}_{p}
\end{align*}
it does not suffice to just check the actual shape of the earth.
We need to consider different (imaginary) possibilities of how the world could have been.
We call these ways the world could have been \emph{possible worlds}.
To determine the truth or falsity of sentences like the above, we need to determine whether certain possible worlds (here: worlds in which the earth is not round) have certain properties, such as whether they are compatible with the rules of logic or with what Alex considers possible.

Many natural language expressions carry a modal meaning component.
In English, there are words like \emph{should}, \emph{could}, \emph{possibly}, \emph{need to}, \emph{be required} and many more.
In order to capture such modal meaning, modal logics provide a formal language with which we can express what is true in the actual world and what is true in other possible worlds.
To do so, modal logics usually have (at least) these two \emph{modal operators}:
\begin{center}
  \begin{tabular}{cll}
    operator   & neutral paraphrase & natural language cues \\ \hline
    $\Diamond$ & It is possible that ... & \emph{might}, \emph{may}, \emph{conceivably}, \dots \\
    $\Box$     & It is necessary that ... & \emph{must}, \emph{have to}, \emph{necessarily}, \dots \\
  \end{tabular}
\end{center}

There are different kinds of modal logic, depending on the kind of modal meaning we would like to capture.
Indeed, there are many different kinds of \emph{modalities}, also referred to as \emph{modal flavors}:
\begin{enumerate}[(i)]
  \item what one ought to do given one's parents' rules and regulations;
  \item what one ought to do given a country's law;
  \item what is logically necessary;
  \item what an agent believes to be true or possible;
  \item[\dots] and more.
\end{enumerate}
Different types of modality require different assumptions about the logical properties of the modal operators.
In this way, we can think of modal logics as tools for \emph{conceptual analysis} of the logical properties that characterize concepts like \emph{belief}, \emph{logical necessity}, \emph{obligation} etc.\sidenote{Saying that modal logics are \emph{tools} for investigating such notions, does not mean that it is uncontroversial that they are the \emph{only} or the \emph{right} tool for this purpose. From a linguistic point of view, formal analyses of the meaning of modal meaning ---like in deontic or epistemic modals, conditionals or imperatives--- builds on ideas from (vanilla) modal logic, as introduced here, but may add more complexity, e.g., \emph{ordering on possible worlds}.}
Here, we will have a look at a modal propositional logic for talking about beliefs and knowledge, a so-called \emph{epistemic modal logic}.

\section{The language of epistemic modal logic}

Modal propositional logic (\modlog) extends propositional logic by including the new operators $\Diamond$ and $\Box$.
To make things more interesting, we will have one pair of operators $\Diamond_{i}$ and $\Box_{i}$ where $i$ is a variable for one of several \emph{agents}.
We will interpret a formula $\Diamond_{i} \varphi$ as: ``agent $i$ considers $\varphi$ to be possible;''
and $\Box_{i}\varphi$ as: ``agent $i$ believes that $\varphi$ is true.''
In this way, we will be able to formally express sentences like ``Alex believes that Bo considers it possible that the earth is not round'' as $\Box_{a} \Diamond_{b} \neg p$.

\subsection{Formulas}

Let $\mathcal{P}$ be a set of proposition letters and let $\mathcal{A}$ be a set of agents.
The language $\mathfrak{L}_{\mathcal{P},\mathcal{A}}$ of \modlog is the set of all \emph{formulas} which are recursively defined as follows:
\begin{enumerate}[(i)]
  \item Every proposition letter is a formula.
  \item If $\varphi$ is a formula, so is $\neg \varphi$, $\Box_{i} \varphi$ and $\Diamond_{i} \varphi$ for each $i \in \mathcal{A}$.
  \item If $\varphi$ and $\psi$ are formulas, so are:
        \vspace*{-0.4cm}
        \begin{multicols}{4}
          \begin{enumerate}[a.]
            \item ($\varphi \wedge \psi$)
            \item ($\varphi \vee \psi$)
            \item ($\varphi \rightarrow \psi$)
            \item ($\varphi \leftrightarrow \psi$)
          \end{enumerate}
        \end{multicols}
        \vspace*{-0.4cm}
  \item Anything that cannot be constructed by (i)--(iii) is not a formula.
\end{enumerate}


Here are examples of well-formed formulas of our multi-agent epistemic logic with paraphrases (where agent $a$ is Alex, and $b$ is Bo):
\begin{center}
  \begin{tabular}{lp{7cm}}
    \vspace{0.3cm}
    $\Box_{a} (p \rightarrow q)$
    & \mygray{Alex believes that if $p$, then $q$.}
    \\
    \vspace{0.3cm}
    $p \rightarrow \Diamond_{a} \Box_{b} q$
    & \mygray{If $p$ is true, then Alex considers it possible that Bo believes that $q$.}
    \\
    $\Box_{a} p \wedge \Box_{b} p \wedge \neg \Box_{a} \Box_{b} p$
    & \mygray{Both Alex and Bo believe that $p$, but Alex doesn't believe that Bo believes it.}

  \end{tabular}
\end{center}

\subsection{Semantics of modal logic}

The semantics of \modlog is defined in terms of so-called \emph{modal models}.
Similar to \proplog and \predlog, a modal model will tell us whether any given formula is true or false.
But while models in \proplog and \predlog were representations of just one possible world, modal models comprise (possibly: infinitely) many possible worlds all at once.

\paragraph{Modal models.} A \emph{modal model} for the language $\mathfrak{L}_{\mathcal{P}, \mathcal{A}}$ is a triple $\Model = \tuple{W,V,(R_i)_{i \in \mathcal{A}}}$ such that:\sidenote{
  Modal models are also often called \emph{Kripke structures}, named after the logician and philosopher of language Saul Kripke.
}
\begin{itemize}[]
  \item $W$ is a set of \emph{(possible) worlds},
  \item $V \ts \colon \ts W \times \mathcal{P}  \ra \set{0,1}$
  is a \emph{valuation function} assigning a unique truth value to every proposition letter for every possible world, and
  \item for each agent $i \in \mathcal{A}$, $R_i \subseteq W
  \times W$ is an \emph{accessibility relation} between possible worlds.
\end{itemize}
Accessibility relations are interpreted as follows: if the actual world is $w$, then agent $i$ considers possible all the worlds that can be ``accessed'' or ``seen'' via relation $R_{i}$ from $w$.
We write $wR_iv$ for $\tuple{w,v} \in R_i$ and introduce the notation:
\begin{align*}
R_i(w) = \set{v \in W \mid wR_iv}
\end{align*}
to refer to the set of \emph{accessible worlds} for agent $i$ from world $w$.

\paragraph{Single-agent example.}
For a simple example, let us just consider a case with a single agent first, which is Alex: $\mathcal{A} = \set{a}$.
We are interested in Alex's beliefs about tomorrow's weather.
For simplicity, we consider three possible states of the weather: it's either rainy, cloudy or sunny.
We can model this in terms of three proposition letters: $\mathcal{P} = \set{r, c, s}$.
For simplicity, let's consider just four possible worlds, $W = \set{w_{1}, w_{2}, w_{3}, w_{4}}$.
We need to define a valuation function that maps every pair consisting of a possible world and a proposition letter onto a truth value.
This could be written like so: $V(w_{1},r) = 1$, $V(w_{1},c) = 0$ and so on.
Here is more a compact representation of the valuation function for this model:\sidenote{
  This looks suspiciously like a truth-table, but it is not.
  Not all logically possible worlds are listed (because we do not need them in this example).
  Also, there are rows ($w_1$ and $w_4$) with exactly the same truth-value assignments (we need them to model different beliefs, see below).
}
\begin{center}
  \begin{tabular}{cccc}
    world & $r$ & $c$ & $s$ \\ \hline
    $w_{1}$ & 1 & 0 & 0 \\
    $w_{2}$ & 0 & 1 & 0 \\
    $w_{3}$ & 0 & 0 & 1 \\
    $w_{4}$ & 1 & 0 & 0 \\
  \end{tabular}
\end{center}
Let's furthermore assume the following accessibility relation:
\begin{align*}
  R_{a} = \set{
  \tuple{w_{1}, w_{2}}, \tuple{w_{1}, w_{3}},
  \tuple{w_{2}, w_{2}}, \tuple{w_{2}, w_{3}},
  \tuple{w_{3}, w_{2}}, \tuple{w_{3}, w_{3}},
  \tuple{w_{4}, w_{4}}
  }
\end{align*}

Figure~\ref{fig:modal-model-single-agent} represents the whole modal model in a much more intelligible way.
The diagram shows, for each possible world, the set of all proposition letters that are true in this world.
Labelled arrows are used to represent the accessibility relation for each agent.
Since we only consider a single agent, all arrows are (superfluously) labelled with $a$.

\begin{figure}
  \centering
  \begin{tikzpicture}
        \tikzset{breakable/.code=\ifpgfmatrix\else\tikzset{matrix of
            nodes}\fi}

        \draw [every
        label/.style={breakable},state/.style={thick,rounded corners,rectangle split,
          rectangle split parts=2, draw=black}]
        (0,0) node[state] (1) {$w_{1}$ \nodepart{second} $V \colon r$}
        (5,0) node[state] (2) {$w_{2}$ \nodepart{second} $V \colon c$}
        (0,-3) node[state] (3) {$w_{3}$ \nodepart{second} $V \colon s$}
        (5,-3) node[state] (4) {$w_{4}$ \nodepart{second} $V \colon r$};

        \draw[->,every node/.style={sloped,above},thick] (1) edge node {$a$} (2);

        % \draw[->,every node/.style={left},thick] (1) edge[loop left] node {$a$} ();

        \draw[->,every node/.style={sloped,above},thick] (1) edge node {$a$} (3);

        \draw[<->,every node/.style={sloped,above},thick] (2) edge node {$a$} (3);

        \draw[->,every node/.style={right},thick] (2) edge[loop right] node {$a$} ();

        \draw[->,every node/.style={below},thick] (3) edge[loop left] node {$a$} ();

        \draw[->,every node/.style={right},thick] (4) edge[loop right] node {$a$} ();

      \end{tikzpicture}
  \caption{Example of a modal model for Alex's beliefs about tomorrow's weather.}
  \label{fig:modal-model-single-agent}
\end{figure}

What beliefs does Alex hold according to the model in Figure~\ref{fig:modal-model-single-agent}? ---
Actually, that depends on which of the four worlds we consider to be the \emph{reference world}, i.e., the world we consider to be actual for the purpose of analysis.
If $w_{1}$ is the reference world (we can also say: ``in world $w_{1}$''), Alex believes that it might be cloudy and that it might be sunny, but that it will not rain.
This is because the worlds accessible from $w_{1}$, according to Alex's accessibility relation, include only the two worlds $w_{2}$ and $w_{3}$ where it is cloudy and sunny respectively: $R_{a}(w_{1}) = \set{w_{2}, w_{3}}$.
So, in world $w_{1}$ Alex \emph{rules out} the possibility of rain (even though that is  the true weather for tomorrow, unbeknownst to Alex); Alex \emph{rules in} the possibilities of cloudy and sunny weather, so that Alex is, after all, not entirely sure about the weather.

In worlds $w_{2}$ and $w_{3}$, Alex holds the exact same beliefs as in $w_{1}$.
This is because we have the same set of accessible worlds for the first three worlds:
\begin{align*}
  R_{a}(w_{1}) = R_{a}(w_{2}) = R_{a}(w_{3}) = \set{w_{2}, w_{3}}
\end{align*}
In $w_{1}$ these beliefs are false ($w_{1}$ is a rain-world, but Alex excludes rain), in $w_{2}$ and $w_{3}$ they are not (e.g., $w_{2}$ is a cloudy world and Alex considers cloudy weather to be possible).

Finally, in world $w_{4}$ Alex's accessible worlds are $R_{a}(w_{4}) = \set{w_{4}}$.
So, in $w_{4}$, Alex entertains only one relevant possibility.
Alex is maximally opinionated, i.e., not uncertain at all.
Alex believes (truthfully) that it will rain.

\paragraph{Truth conditions for \modlog.}
Truth conditions for formulas of \modlog are defined relative to \emph{pointed models}.
If $\Model = \tuple{W,V,(R_a)_{a \in \mathcal{A}}}$ and $w \in W$, the pair $\Model$, $w$ is a pointed model.
A valuation function $V_{\Model, w}$ for a pointed model, assigns truth values to each formula of \modlog as follows:
For proposition letters $p \in \mathcal{P}$, the model's internal valuation function $V$ decides on truth or falsity:\sidenote[][-2cm]{Notice that the function $V_{\Model, w}$ on the left-hand side is the ``global'' valuation function that builds on the pointed model to assign truth values to \emph{all} formulas, while the valuation function $V$ on the right-hand side is the valuation function \emph{inside} of model $\Model$ gives truth values to \emph{only} proposition letters (for a given world).}
%
\begin{align*}
  V_{\Model, w}(p) = 1 & \text{ \ iff \  } V(w,p) = 1
\end{align*}
%
Formulas with main operators from propositional logic are treated as before:
%
\begin{align*}
 & V_{\Model,w}(\neg \phi) = 1                 & & \text{ \ iff \ } V_{\Model,w}(\phi) =0\\
 & V_{\Model,w}(\phi \wedge \psi) = 1          & & \text{ \ iff \ } V_{\Model,w}(\phi) =1 \text{ and }  V_{\Model,w}(\psi) = 1\\
 & V_{\Model,w}(\phi \vee \psi) = 1            & & \text{ \ iff \ } V_{\Model,w}(\phi) =1 \text{ or  }  V_{\Model,w}(\psi) = 1\\
 & V_{\Model,w}(\phi \rightarrow \psi) = 0     & & \text{ \ iff \ } V_{\Model,w}(\phi) =1 \text{ and }  V_{\Model,w}(\psi) = 0\\
 & V_{\Model,w}(\phi \leftrightarrow \psi) = 1 & & \text{ \ iff \ } V_{\Model,w}(\phi) = V_{\Model,w}(\psi)
\end{align*}

\noindent What is new is the treatment of modal operators.
For these, the accessibility relations are important.\sidenote{
  Essentially, modal operators are like quantifiers from predicate logic, but quantifying over ``accessible worlds.''
}
%
\begin{align*}
    & V_{\Model, w}(\Box_i \phi) = 1 & & \text{ \ iff \ } V_{\Model, v}(\phi) = 1 \text{ for all $v \in R_i(w)$} \\
    & V_{\Model, w}(\Diamond_i \phi) = 1 & & \text{ \ iff \ } V_{\Model, v}(\phi) = 1 \text{ for some $v \in R_i(w)$}
\end{align*}
%
Crucially, the semantics for modal operators shifts the reference world, so to speak, switching from pointed model $\Model, w$ to $\Model, v$ for some accessible world $v \in R_{i}(w)$.
In this way, we can also give a meaning to nested modal operators, as the following example demonstrates.

\paragraph{A more complex example.}
Figure~\ref{fig:modal-model} shows an example of a modal model for two agents ($\mathcal{A} = \set{a,b}$, Alex and Bo) and just two propositions letters ($\mathcal{P} = \set{p, q}$).
This model only has three possible worlds $W = \set{w,u,v}$.
The diagram shows the set of all proposition letters that are true in each world.
Labelled arrows are used to represent the accessibility relation for each agent.

\begin{figure*}
  % \centering
  \begin{minipage}{0.4\linewidth}
    \begin{itemize}[]
      \item $\Bel_b (\neg p \wedge \neg q)$  \\ \mygray{``Bo believes that $p$ and $q$ are false''}
      \item $\neg p \wedge \Bel_a p$  \\ \mygray{``Alex falsely believes that $p$ is true''}
      \item $\neg \Bel_a q \wedge \neg \Bel_a \neg q$ \\ \mygray{``Alex is uncertain about $q$''}
      \item $\neg \Bel_b p \wedge \Bel_a \Bel_b p$ \\ \mygray{``Alex falsely believes that Bo believes $p$''}
      \item $\neg \Bel_a \Bel_b q \wedge \neg \Bel_a \neg \Bel_b q$ \\ \mygray{``Alex is uncertain whether Bo believes $q$''}
      \item $\Bel_a ((q \ra \Bel_b q) \wedge ( \neg q \ra \Bel_b \neg q ))$ \\ \mygray{``Alex believes that Bo knows whether $q$''}
      \item[]
    \end{itemize}
  \end{minipage}
  \begin{minipage}{0.4\linewidth}
    \begin{tikzpicture}
      \tikzset{breakable/.code=\ifpgfmatrix\else\tikzset{matrix of
          nodes}\fi}

      \draw [every
      label/.style={breakable},state/.style={thick,rounded
        corners,rectangle split, rectangle split parts=2,
        draw=black}] (0,0) node[state] (w) {$w$ \nodepart{second} $V
        \colon \emptyset$} (5,0) node[state] (v) {$v$
        \nodepart{second} $V \colon p,q$} (2.5,-3.3) node[state] (u)
      {$u$ \nodepart{second} $V \colon p$};

      \draw[->,every node/.style={sloped,above},thick] (w) edge node
      {$a$} (v);

      \draw[->,every node/.style={left},thick] (w) edge[loop left]
      node {$b$} ();

      \draw[<->,every node/.style={sloped,above},thick] (v) edge
      node {$a$} (u);

      \draw[->,every node/.style={right},thick] (v) edge[loop right]
      node {$a,b$} ();

      \draw[->,every node/.style={sloped,above},thick] (w) edge node
      {$a$} (u);

      \draw[->,every node/.style={below},thick] (u) edge[loop below]
      node {$a,b$} ();

    \end{tikzpicture}
  \end{minipage}
  \caption{
    Example of a modal model for two agents.
    }
  \label{fig:modal-model}
\end{figure*}

The formulas on the left-hand side of Figure~\ref{fig:modal-model} are all true in world $w$.
To check that, for example, $V_{\Model, w}(\neg \Bel_a \Bel_b q \wedge \neg \Bel_a \neg \Bel_b q) = 1$, we can reason as follows:
\begin{align*}
  & V_{\Model, w}(\neg \Bel_a \Bel_b q \wedge \neg \Bel_a \neg \Bel_b q) = 1 \\
  \text{ iff \ \ \ } & V_{\Model, w}(\neg \Bel_a \Bel_b q) = 1 \text {\ \ \  and \ \ \ }
                       V_{\Model, w}(\neg \Bel_a \neg \Bel_b q) = 1
\end{align*}
Consider the first conjunct:
\begin{align*}
  & V_{\Model, w}(\neg \Bel_a \Bel_b q) = 1 \\
  \text{ iff \ \ \ } & \text{for some } w' \in R_{a}(w) \  V_{\Model, w'}(\Bel_b q) = 0 \\
  \text{ iff \ \ \ } & \text{for some } w' \in R_{a}(w') \ \text{for some } w'' \in R_{b}(w') \  V_{\Model, w''}(q) = 1 \\
  & \text{which is true if we set $w' = w'' = v$}
\end{align*}
Consider the second conjunct:
\begin{align*}
  & V_{\Model, w}(\neg \Bel_a \neg \Bel_b q) = 1 \\
  \text{ iff \ \ \ } & \text { for some } w' \in R_{a}(w)\  V_{\Model, w'}(\neg \Bel_b q) = 0 \\
  \text{ iff \ \ \ } & \text { for some } w' \in R_{a}(w)\  V_{\Model, w'}(\Bel_b q) = 1 \\
  \text{ iff \ \ \ } & \text { for some } w' \in R_{a}(w)\ \text{for all } w'' \in R_{b}(w') \  V_{\Model, w'}(q) = 1 \\
  & \text{which is true if $w' = u$, since $R_{b}(u) = \set{u}$ and $q$ is false in $u$}
\end{align*}

\subsection{Validity, entailment etc.}

The notions of validity, entailment, tautology, contradiction and contingency are defined essentially in the same way as for \proplog and \predlog.
We only need to talk about truth at a pointed model.
For example, we say that $\varphi$ is a tautology of \modlog whenever $V_{\Model, w}(\varphi)=1$ for all pointed models $\Model, w$.

However, depending on the kind of modal concept we would like to capture, different modal logics might want to place additional restrictions on which pointed models they consider to begin with.
Especially, it is common to impose additional constraints on the nature of the accessibility relation.
This is the topic of the next section, which investigates constraints on the accessibility relation for consistent and introspective beliefs.
As a result, we actually have multiple notions of validity, entailment and so on.
For example, we would say that $\varphi$ is a tautology in the \modlog of consistent and introspective beliefs (see next section) whenever $V_{\Model, w}(\varphi)=1$ for all pointed models $\Model, w$ which satisfies the additional requirements we impose to guarantee consistent and introspective beliefs.

\section{The logic of rational belief}

So far, we have not put any constraints on the accessibility relations in our modal models.
But, in order to represent beliefs that are ``rational'' in a crucial sense, we should.
For example, we should rule out that $R_{i}(w) = \emptyset$ for some agent $i$ and world $w$, because that would mean that agent $i$ believes \emph{anything} at this world, including logical contradictions (which isn't very rational, right?).
This is to say that we would require rational agents to hold \emph{consistent beliefs} that are contradiction-free.
We can ensure contradiction-freedom by a constraint on admissible modal models, called \emph{seriality}:
\begin{itemize}[]
  \item \textbf{Seriality:} \ \  for all $i$ and $w$: \ $R_{i}(w) \neq \emptyset$
\end{itemize}
or, in other words, there is always at least one accessible world.
We can then prove the following:

\begin{claim}
  If a modal model's accessibility relation is serial, no agent can believe in a contradiction.
\end{claim}
\begin{proof}
  Let $\Model, w$ be a pointed model with $R_{i}(w)$ the accessible worlds for agent $i$ at $w$.
  Take an arbitrary contradiction $\varphi$.
  Being a contradiction entails that $V_{\Model', v}(\varphi) = 0$ for all $\Model'$ and worlds $v$.
  Now assume towards contradiction that $i$ believes $\varphi$ in $w$ for $\Model$, which means that $V_{\Model, w}(\Box_{i}\varphi)=1$.
  This entails that for all $v \in R_{a}(w)$ we have $V_{\Model, v}(\varphi) = 1$.
  By seriality, there must therefore be at least one world $v \in R_{a}(w)$ with $V_{\Model, v}(\varphi) = 1$.
  But there cannot be any world that makes $\varphi$ true, with $\varphi$ being a contradiction.
  Thus, our assumption that $i$ believes $\varphi$ in $w$ for $\Model$ cannot be true.
\end{proof}

Moreover, we might want to put additional constraints on higher-order beliefs of an agent.
A higher-order belief of agent $i$ is a belief of $i$ about their own beliefs, e.g., that Alex believes that Alex considers it possible that $\varphi$ ($\Box_{a}\Diamond_{a}\varphi$).
A common requirement for rational higher-order beliefs is \emph{positive introspection}: if Alex believes that $\varphi$, then Alex believes that they believe $\varphi$.
Positive introspection is guaranteed by transitive accessibility relations.
\begin{itemize}[]
  \item \textbf{Positive Introspection:} \ \  for all $i$: \ $R_{i}$ is transitive
\end{itemize}
\begin{claim}
 If an agent $i$'s accessibility relation is transitive in model $\Model$, then $V_{\Model,w}(\Box_{i}\varphi \rightarrow \Box_{i}\Box_{i}\varphi) = 1$ for all $w$ in that model.
\end{claim}
\begin{proof}
  Let $\Model$ have a transitive accessibility relation $R_{i}$.
  Suppose towards contradiction that $V_{\Model,w}(\Box_{i}\varphi \rightarrow \Box_{i}\Box_{i}\varphi) = 0$ at some world $w$.
  That can only be the case if $V_{\Model,w}(\Box_{i}\varphi) = 1$ and $V_{\Model,w}(\Box_{i}\Box_{i}\varphi) = 0$.
  From $V_{\Model,w}(\Box_{i}\Box_{i}\varphi) = 0$ we infer that there is a $w' \in R_{i}(w)$ such that there is some $w'' \in R_{i}(w')$ such that $V_{\Model, w''}(\varphi) = 0$.
  Since, by $R_{i}$ is assumed to be transitive, we derive that $w'' \in R_{i}(w)$.
  But that contradicts $V_{\Model,w}(\Box_{i}\varphi) = 1$, which we have derived previously.
\end{proof}

A final requirement on rational higher-order beliefs, which is often made but more controversial, is that rational agents should also have \emph{negative introspection}: if Alex does not believe that $\varphi$, then Alex believes that they don't believe $\varphi$.
Negative introspection is guaranteed by Euclidean accessibility relations.
A relation $R \subseteq W \times W$ is Euclidean iff for all $w,u,v \in W$: if $wRu$ and $wRv$, then $uRv$.
\begin{itemize}[]
  \item \textbf{Negative Introspection:} \ \  for all $i$: \ $R_{i}$ is Euclidean
\end{itemize}
\begin{claim}
 If an agent $i$'s accessibility relation is Euclidean in model $\Model$, then $V_{\Model,w}(\neg \Box_{i}\varphi \rightarrow \Box_{i} \neg \Box_{i}\varphi) = 1$ for all $w$ in that model.
\end{claim}
\begin{proof}
  Let $\Model$ have a Euclidean accessibility relation $R_{i}$.
  Suppose towards contradiction that $V_{\Model,w}(\neg \Box_{i}\varphi \rightarrow \Box_{i} \neg \Box_{i}\varphi) = 0$ at some world $w$.
  That can only be the case if $V_{\Model,w}( \neg \Box_{i}\varphi) = 1$ and $V_{\Model,w}( \neg \Box_{i}\Box_{i}\varphi) = 0$.
  From the former we learn that there is a world $w^{*} \in R_{i}(w)$ such that $V_{\Model, w^{*}}(\varphi)=0$.
  From the latter we learn that for all world $w' \in R_{i}(w)$ it holds that $V_{\Model,w'}(\Box_{i}\varphi) = 1$.
  But if the relation is Euclidean, then any world $w' \in R_{i}(w)$ must also ``see'' $w^{*}$, so that is a contradiction.
\end{proof}

We say that an agent's beliefs are consistent and introspective if all three conditions (seriality, positive \& negative introspection hold).
We can summarize the requirements of ``rational'' beliefs more succinctly, because it follows from the above definitions that agent $i$'s beliefs are consistent and introspective iff for all $w \in W$:
\begin{enumerate}[(i)]
  \item $R_{i}(w) \neq \emptyset$
  \item $R_{i}(w) = R_{i}(v)$ for all $v \in R_{i}(w)$.
\end{enumerate}
Intuitively, for all possible worlds, there is at least one world agent $i$ considers possible, and the worlds accessible by agent $i$ form a ``fully connected cloud'' in which every world has access to all and only worlds in that cloud.\sidenote{We could have given this simpler definition of ``rational'' beliefs in the first place. But this would not allow us to systematically study deviations from this norm of rationality of beliefs. Formulating constraints in terms of properties of accessibility relations give the logician more grip to tune her system of modal logic in exactly the way she wants it to be. This is similar to the granularity and variability we get for systems of logical reasoning defined in terms of natural deduction rules. Indeed, in a natural deduction system for modal logic, we would see deduction rules that correspond directly to certain properties of the accessibility relations.}


\bigskip
\noindent \colorbox{mygray}{\centering
  \begin{minipage}{1.0\textwidth}

    \begin{exercise}
      Translate the following sentences into epistemic modal logic.
      Use as your translation key: $p$ ``the earth is round'', $q$ ``the moon is made of cheese'', ``a'' Alex, and $b$ ``Bo.''
      \begin{enumerate}[(i)]
        \item If Alex believes that the moon is made of cheese, then Alex also believes that the earth is not round.
        \item Bo believes that Alex thinks it's possible that Bo believes that the moon is made of cheese.
        \item Alex considers it possible that, if the moon is made of cheese, then Bo believes it.
        \item Whatever Bo believes about whether the earth is round, Alex believes the same.
        \item Alex is uncertain about whether Bo believes that Alex believes that the moon is made of cheese.
      \end{enumerate}
    \end{exercise}

    \begin{exercise}
      Here is a full mathematical specification of a modal logical language and a modal model $\Model$:

      \begin{itemize}[]
        \item $\mathcal{P} = \set{p,q}$, \ \ $\mathcal{A} = \set{a, b}$, \ \  $W = \set{w_1, w_{2}, w_{3}, w_{4}}$
        \item $V(w_{1},p) = V(w_{1},q) = 1$, \ \ $V(w_{2},p) =1,  V(w_{2},q) = 0$
        \item  $V(w_{3},p) =0,  V(w_{3},q) = 1$, \ \ $V(w_{4},p) =1,  V(w_{4},q) = 1$
        \item $R_{a} = \set{
        \tuple{w_{1}, w_{2}},
        \tuple{w_{2}, w_{3}},
        \tuple{w_{4}, w_{3}}
        }$
        \item $R_{b} = \set{
        \tuple{w_{1},w_{4}},
        \tuple{w_{2}, w_{4}},
        \tuple{w_{3}, w_{4}},
        \tuple{w_{4},w_{1}}, \tuple{w_{4},w_{4}}
        }$
      \end{itemize}
      \begin{enumerate}
        \item Draw the model as a diagram like in Figure~\ref{fig:modal-model}.
        \item Determine the truth value for the following:
        \vspace{-0.3cm}
        \begin{multicols}{2}
          \begin{enumerate}[a]
            \item $V_{\Model, w_{1}}(\Box_{a} p)$
            \item $V_{\Model, w_{1}}(\Box_{a} p \wedge \Box_{a} \Diamond_{a} \neg p)$
            \item $V_{\Model, w_{3}}(\Box_{a} (p \wedge \neg p))$
            \item $V_{\Model, w_{4}}(p \rightarrow \Box_{b} p )$
          \end{enumerate}
        \end{multicols}
        \item Does this model satisfy the constraints on rational (higher-order) beliefs?
        What minimal changes to the accessibility relations would make it satisfy these constraints?
        Which of the previous truth-values in pointed models change when you look at rational beliefs?
      \end{enumerate}
    \end{exercise}

  \end{minipage}
}

\newpage

\section{Excursion: Applications of modal logic in linguistics and beyond}

Beyond a formalization of (rational) beliefs and means to fix the truth-conditions for statements about them, epistemic modal logics have many other applications for linguistics and other neighboring fields.

\subsection{Strict conditionals}

The truth-conditional semantics for the material conditional $p \rightarrow q$ does not square very well with common intuitions about the meaning of natural language ``if \dots, then \dots'' sentences.
If Jones says ``It's false that if $p$, then $q$,'' they are not saying that $p$ is true and $q$ is false, but that is exactly what the truth-table for $p \rightarrow q$ would give us.

An alternative analysis for natural language conditionals, which is maybe better (but still not unproblematic), is that of the \emph{strict conditional}.
According to a strict conditional analysis, we would translate ``if $p$, then $q$'' to something like $\Box_{i}(p \rightarrow q)$, where $i$ is the speaker.
So, we would treat a conditional sentence as (possibly: implicitly) modalized; a statement about the ways the world could be that the speaker considers possible.
If Jones says ``It's false that if $p$, then $q$,'' this would be translated as $\neg \Box_{j}(p \rightarrow q)$, which is much weaker than before, as it only requires that Jones thinks that it is conceivable that $p$ is true but $q$ is false.

\subsection{Common ground}

In communication, we must keep track of who knows what.
Otherwise, we risk redundancy and boredom or miscommunication and frustration.
Leading theories of pragmatic language use therefore assume that interlocutors keep track of a \emph{common ground} which (among other things) contains the shared beliefs of all participants in a conversation.
Modal logics of belief help define such inter-personal notions like \emph{mutual belief} or \emph{common belief}.


% \paragraph{Notation and terminology.}

% In everything that follows, we implicitly restrict attention to what are called \emph{serial models}, where $R_i(w) \neq \emptyset$ for all $w$ and $i$.
% We say that a model is reflexive ($r$), transitive ($t$) or Euclidean ($e$) if all of its relations $R_i$ are reflexive, transitive or Euclidean.
% For $C \subseteq \set{r,t,e}$ a set of properties, let $\Models^C$ refer to the collection of all models $\Model_{\mathcal{P},\mathcal{A}}$ that have the properties selected in $C$.
% Specifically:
% \begin{itemize}[]
%   \item $\Models^{\set{t,e}}$ are \emph{belief models}, and
%   \item $\Models^{\set{r,t,e}}$ are \emph{knowledge models}.
% \end{itemize}


% \subsection{Semantics}
% \label{sec:semantics}



% \begin{definition}[Truth \& Validity]
%   Fix $\Model = \tuple{W,V,(R_j)_{j \le i}}$ and $w \in W$ and define \textbf{truth in
%   pointed models} inductively:

% \vspace{-1em}
% \begin{minipage}{0.45\linewidth}
%   \begin{align*}
%     % \Model, w \models \bot & \text{ \  never}\\
%     % \Model, w \models \top & \text{ \ always}\\
%     \Model, w \models p & \text{ \ iff \  } V(w,p) = 1\\
%     \Model, w \models \neg \phi & \text{ \ iff \ } \Model, w \not \models \phi
%   \end{align*}
% \end{minipage}
% \begin{minipage}{0.45\linewidth}
%   \begin{align*}
%     % \Model, w \models \phi \wedge \psi & \text{ \ iff \ } \Model, w
%     % \models \phi \text{ \ and \ } \Model, w \models \psi \\
%     \Model, w \models \phi \wedge \psi & \text{ \ iff \ } \Model, w
%     \models \phi \text{ \ and \ } \Model, w \models \psi \\
%     % \Model, w \models \phi \ra \psi & \text{ \ iff \ if \ } \Model,
%     % w
%     % \models \phi \text{ \ then \ } \Model, w \models \psi \\
%     % \Model, w \models \phi \leftrightarrow \psi & \text{ \ iff \ }
%     % \Model, w
%     % \models \phi \text{ \ if and only if \ } \Model, w \models \psi
%     % \\
%     \Model, w \models \Bel_i \phi & \text{ \ iff \ } \Model, v \models
%     \phi \text{ \ for all $v \in R_i(w)$}
%     % \Model, w \models \Poss_i \phi & \text{ \ iff \ } \Model, v
%     % \models \phi \text{ \ for some $v \in R_i(w)$}
%   \end{align*}
% \end{minipage}
% \medskip \medskip

%  We say that a formula $\phi$ is \textbf{valid in a model} $\Model$,
%   $\Model \models \phi$, if for all $w \in W$ of that model $\Model, w
%   \models \phi$. We say that a formula $\phi$ is \textbf{valid in a
%     class of models} $\mathcal{N}$, $\mathcal{N} \models \phi$, if
%   $\phi$ is valid on every model in $\mathcal{N}$.
% \end{definition}


% \begin{example}[Modal Model]
%   \label{example:simple_modal_modal}
%   We consider a simple modal model for $\Props = \set{p,q}$ and two
%   agents $i$ and $j$. The following formulas are true at world $w$:

%   \begin{minipage}[t]{0.4\linewidth}
%     \vspace{0em}
%     \begin{itemize}
%     \item $\Bel_j (\neg p \wedge \neg q)$  \\ ``$j$ believes that $p$ and $q$ are false''
%     \item $\neg p \wedge \Bel_i p$  \\ ``$i$ falsely believes that $p$ is true''
%     \item $\neg \Bel_i q \wedge \neg \Bel_i \neg q$ \\ ``$i$ is uncertain about $q$''
%     \item $\neg \Bel_j p \wedge \Bel_i \Bel_j p$ \\ ``$i$ falsely believes that $j$ believes $p$''
%     \item $\neg \Bel_i \Bel_j q \wedge \neg \Bel_i \neg \Bel_j q$ \\ ``$i$ is uncertain whether $j$ believes $q$''
%     \item $\Bel_i (q \ra \Bel_j q \wedge \neg q \ra \Bel_j \neg q)$ \\ ``$i$ believes that $j$ knows whether $q$''
%     \end{itemize}
%   \end{minipage}
%   \begin{minipage}[t]{0.6\linewidth}
%     \vspace{0em}
%     \begin{center}
%       \begin{tikzpicture}
%         \tikzset{breakable/.code=\ifpgfmatrix\else\tikzset{matrix of
%             nodes}\fi}

%         \draw [every
%         label/.style={breakable},state/.style={thick,rounded
%           corners,rectangle split, rectangle split parts=2,
%           draw=black}] (0,0) node[state] (w) {$w$ \nodepart{second} $V
%           \colon \emptyset$} (5,0) node[state] (v) {$v$
%           \nodepart{second} $V \colon p,q$} (2.5,-3.3) node[state] (u)
%         {$u$ \nodepart{second} $V \colon p$};

%         \draw[->,every node/.style={sloped,above},thick] (w) edge node
%         {$i$} (v);

%         \draw[->,every node/.style={left},thick] (w) edge[loop left]
%         node {$j$} ();

%         \draw[<->,every node/.style={sloped,above},thick] (v) edge
%         node {$i$} (u);

%         \draw[->,every node/.style={right},thick] (v) edge[loop right]
%         node {$i,j$} ();

%         \draw[->,every node/.style={sloped,above},thick] (w) edge node
%         {$i$} (u);

%         \draw[->,every node/.style={below},thick] (u) edge[loop below]
%         node {$i,j$} ();

%       \end{tikzpicture}
%     \end{center}
%   \end{minipage}

% \end{example}


% \subsection{Proof System}
% \label{sec:proof-system}

% \begin{definition}[Proof System for Modal Logics]
%   A proof system for a modal logical language is given by a set of $I$
%   \textbf{inference rules} and a set $A$ of \textbf{axiom
%     schemas}. Let $I$ contain the rules:
%   \begin{itemize}
%   \item[\textbf{MP}:] from $\phi$ and $\phi \ra \psi$ infer $\psi$ \hfill (\textbf{modus ponens})
%   \item[\textbf{Nec}:] from $\phi$ infer $\Bel_i \phi$ \hfill (\textbf{necessitation})
%   \end{itemize}
%   and let $A$ always contain the \textbf{normal axiom schemata}:
%   \begin{itemize}
%   \item[\textbf{Prop}:] all substitution instances of tautologies of propositional logic
%   \item[\textbf{K}:] $(\Bel_i \phi \wedge \Bel_i (\phi \ra \psi)) \ra
%     \Bel_i \psi$
%   \end{itemize}
%   and additionally any (possibly empty) subset of the following \textbf{additional axiom schemata}:
%   \begin{itemize}
%   \item[\textbf{T}:] $\Bel_i \phi \ra \phi$ \hfill (\textbf{truth})
%   \item[\textbf{4}:] $\Bel_i \phi \ra \Bel_i \Bel_i \phi$ \hfill
%     (\textbf{positive introspection})
%   \item[\textbf{5}:] $\neg \Bel_i \phi \ra \Bel_i \neg \Bel_i \phi$ \hfill
%     (\textbf{negative introspection})
%   \end{itemize}
% \end{definition}

% \begin{definition}[Proof System, Proof]
%   A \textbf{proof} is a sequence of formulas each of which is either
%   an axiom or derived from previous formulas in the sequence by
%   application of an inference rule. For any (possibly empty)
%   subset $D$ of the additional axioms $\set{\mathbf{T},
%     \mathbf{4},\mathbf{5}}$ we write $D \vdash \phi$ if a proof exists
%   for $\phi$ in the corresponding proof system.
% \end{definition}

% \paragraph{Soundness \& Completeness}
% \label{sec:soundn-compl}

% \begin{theorem}[Soundness \& Completeness]
%   Let $C$ be a (possibly empty) subset of $\set{r,t,e}$ and let $D$ the
%   `corresponding' subset of $\set{\mathbf{T},
%     \mathbf{4},\mathbf{5}}$. Then:
%   \begin{align*}
%     \Models^C \models \phi \ \ \text{iff} \ \ D \vdash \phi\,.
%   \end{align*}
%   The ``left-to-right'' part is \textbf{completeness}, the ``right-to-left''
%   part is \textbf{soundness}.
% \end{theorem}


\end{document}
