\PassOptionsToPackage{table}{xcolor}
\documentclass[nobib,nofonts]{tufte-handout}

%\geometry{showframe} % display margins for debugging page layout

%%% MF additions
% \usepackage[table]{xcolor}
\usepackage[nographicx, nohyperref, nosubcaption, nogb4e, nobiblatex]{../99-auxiliary-files/00-mypackages}
\usepackage{../99-auxiliary-files/00-mycommands}
\usepackage{../99-auxiliary-files/00-myenvironments}

\usepackage{titlesec}
\usepackage{etoolbox}
\usepackage{tikz-qtree}
\usepackage{subcaption}

% \titleformat{\section}
% {\large\bfshape}{\thesection}{1em}{}

\setcounter{secnumdepth}{5}
\renewcommand\thesection{\arabic{section}}

% this length controls tha hanging indent for titles
% change the value according to your needs
\newlength\titleindent
\setlength\titleindent{0.7cm}

\pretocmd{\paragraph}{\stepcounter{subsection}}{}{}
\pretocmd{\subparagraph}{\stepcounter{subsubsection}}{}{}

\titleformat{\chapter}[block]
  {\normalfont\huge\bfseries}{}{0pt}{\hspace*{-\titleindent}}

\titleformat{\section}
  {\normalfont\Large\itshape}{\llap{\parbox{\titleindent}{\thesection\hfill}}}{0em}{}

\titleformat{\subsection}
  {\normalfont\itshape}{\llap{\parbox{\titleindent}{\thesubsection\hfill}}}{0em}{}

\titleformat{\subsubsection}
  {\normalfont\normalsize\itshape}{\llap{\parbox{\titleindent}{\thesubsubsection}}}{0em}{}

\titleformat{\paragraph}[runin]
  {\normalfont\normalsize\itshape}{}{-0.7cm}{}[\xspace \ \ \ \ ]

\titleformat{\subparagraph}[runin]
  {\normalfont\normalsize}{\llap{\parbox{\titleindent}{\thesubsubsection\hfill}}}{0em}{}

\titlespacing*{\chapter}{0pt}{0pt}{20pt}
\titlespacing*{\subsubsection}{0pt}{3.25ex plus 1ex minus .2ex}{1.5ex plus .2ex}
\titlespacing*{\paragraph}{0pt}{3.25ex plus 1ex minus .2ex}{0em}
\titlespacing*{\subparagraph}{0pt}{3.25ex plus 1ex minus .2ex}{0em}

\DefineNamedColor{named}{mygray2}{cmyk}{0.55,0.25,0.25,0.25}
\newcommand{\mygray}[1]{\textcolor{mygray2}{#1}}

%%% Tufte style
\usepackage{graphicx} % allow embedded images
  \setkeys{Gin}{width=\linewidth,totalheight=\textheight,keepaspectratio}
  \graphicspath{{graphics/}} % set of paths to search for images

\usepackage{fancyvrb} % extended verbatim environments
  \fvset{fontsize=\normalsize}% default font size for fancy-verbatim environments

% Standardize command font styles and environments
\newcommand{\doccmd}[1]{\texttt{\textbackslash#1}}% command name -- adds backslash automatically
\newcommand{\docopt}[1]{\ensuremath{\langle}\textrm{\textit{#1}}\ensuremath{\rangle}}% optional command argument
\newcommand{\docarg}[1]{\textrm{\textit{#1}}}% (required) command argument
\newcommand{\docenv}[1]{\textsf{#1}}% environment name
\newcommand{\docpkg}[1]{\texttt{#1}}% package name
\newcommand{\doccls}[1]{\texttt{#1}}% document class name
\newcommand{\docclsopt}[1]{\texttt{#1}}% document class option name
\newenvironment{docspec}{\begin{quote}\noindent}{\end{quote}}% command specification environment

\newcommand{\proplog}{\acro{PropLog}}

%%%%%%%%%%%%%%%%%%%%%%%%%%%%%%%%%%%%%%%%%%%%%%%%%%

% \usepackage[sc,osf]{mathpazo}
% \linespread{1.05}



\title{Natural deduction for propositional logic}

\author[M.~Franke]{Michael Franke}

\date{} % without \date command, current date is supplied

\begin{document}

\maketitle

\begin{abstract}
\noindent
natural deduction; soundness \& completeness
\end{abstract}

\begin{itemize}
  \item derivation (proof): chain of legitimate rewrite steps
  \item legitimate steps are: introduction and elimination of connectives
\end{itemize}

A derivation is a finite set of formulas


\newpage

\subsection{Introduction rule for conjunction $I_{\wedge}$}

We may introduce the conjunction $\varphi \wedge \psi$ whenever both the conjuncts $\varphi$ and $\psi$ are available at previous lines $m_{1}$ and $m_{2}$. It does not matter whether $m_{1}$ occurs before $m_{2}$ or the other way around.\sidenote{We adopt the same convention of omitting the outermost parentheses. Strictly speaking, we should write $(\varphi \wedge \psi)$ in line n. of this derivation.}

\bigskip
\noindent \colorbox{mygray!60}{\centering
  \begin{minipage}[t]{0.35\linewidth}
    \textbf{Conjunction Intro $I_{\wedge}$}
  \end{minipage}
  \begin{minipage}[t]{0.55\linewidth}
    \begin{tabular}{clcl}
      $\vdots$ & $\vdots$ & \\
      m$_{1}$ & $\varphi$  &  \\
      $\vdots$ & $\vdots$ & \\
      m$_{2}$ & $\psi$ & \\
      $\vdots$ & $\vdots$ & \\
      n. & $\varphi \wedge \psi$ & & $I_{\wedge}$, m$_{1}$, m$_{2}$
    \end{tabular}
  \end{minipage}
}
\bigskip

We can use this rule to show that $p, q, r \vdash (r \wedge p) \wedge q$ like so:

\begin{tabular}{clcl}
  1. & $p$ & & ass. \\
  2. & $q$ & & ass. \\
  3. & $r$ & & ass. \\
  4. & $r \wedge p$ & & $I_{\wedge}$, 3, 1  \\
  5. & $(r \wedge p) \wedge q$ & & $I_{\wedge}$, 4, 2  \\
\end{tabular}


\subsection{Elimination rule for conjunction $E_{\wedge}$}

If we have the conjunction $\varphi \wedge \psi$, we are allowed to also derive each conjunct.\sidenote{It is not necessary to derive both, we can also only derive one of the disjuncts.}

\bigskip
\noindent \colorbox{mygray!60}{\centering
  \begin{minipage}[t]{0.35\linewidth}
    \textbf{Conjunction Elim $E_{\wedge}$}
  \end{minipage}
  \begin{minipage}[t]{0.55\linewidth}
    \begin{tabular}{clcl}
      $\vdots$ & $\vdots$              & \\
      m        & $\varphi \wedge \psi$ &  \\
      $\vdots$ & $\vdots$              & \\
      n$_{1}$  & $\varphi$             & & $E_{\wedge}$, m \\
      n$_{2}$  & $\psi$                & & $E_{\wedge}$, m
    \end{tabular}
  \end{minipage}
}
\bigskip

We can use this new rule to show that $p \wedge q \vdash q \wedge p$ like so:

\begin{tabular}{clcl}
  1. & $p \wedge q$ & & ass. \\
  2. & $p$          & & $E_{\wedge}$, 1 \\
  3. & $q$          & & $E_{\wedge}$, 1 \\
  4. & $q \wedge p$ & & $I_{\wedge}$, 3, 2
\end{tabular}


\subsection{Elimination rule for implication $E_{\rightarrow}$}

If we have $\varphi \rightarrow \psi$ and $\varphi$ somewhere in our derivation (no matter which one comes first), we can derive $\psi$.

\bigskip
\noindent \colorbox{mygray!60}{\centering
  \begin{minipage}[t]{0.35\linewidth}
    \textbf{Implication Elim $E_{\rightarrow}$}
  \end{minipage}
  \begin{minipage}[t]{0.55\linewidth}
    \begin{tabular}{clcl}
      $\vdots$ & $\vdots$                   & \\
      m$_{1}$  & $\varphi \rightarrow \psi$ &  \\
      $\vdots$ & $\vdots$                   & \\
      m$_{2}$  & $\varphi$                  &  \\
      $\vdots$ & $\vdots$                   & \\
      n        & $\psi$                     & & $E_{\rightarrow}$, m$_{1}$, m$_{2}$
    \end{tabular}
  \end{minipage}
}
\bigskip

Using this rule, we can show that $p \wedge r, r \rightarrow q \vdash p \wedge q$:

\begin{tabular}{clcl}
  1. & $p \wedge r$      & & ass. \\
  2. & $r \rightarrow q$ & & ass.  \\
  3. & $p$               & & $E_{\wedge}$, 1  \\
  4. & $r$               & & $E_{\wedge}$, 1  \\
  5. & $q$               & & $E_{\rightarrow}$, 2, 4 \\
  6. & $p \wedge q$      & & $I_{\wedge}$, 3, 5
\end{tabular}

\subsection{Introduction rule for implication $I_{\rightarrow}$}

The introduction rule for implication is slightly more complex.
The idea is this.
We can introduce $\varphi \rightarrow \psi$ if it is possible to derive $\psi$ from the additional assumption that $\varphi$.
We therefore allow \emph{additional, temporary assumptions} to be introduced in order to make ``thought experiments'' like imagining that some formula was given as well.
We use special notation to note where such an additional assumption was made and where this assumption is dropped again.\sidenote{Notice that we do not need to write down which previous lines this rule operates on as this is implicit in the notation used for marking the ``thought experiment'' or better put: the scope of the additional assumption.}

\bigskip
\noindent \colorbox{mygray!60}{\centering
  \begin{minipage}[t]{0.35\linewidth}
    \textbf{Implication Elim $E_{\rightarrow}$}
  \end{minipage}
  \begin{minipage}[t]{0.55\linewidth}
    \begin{tabular}{cclcl}
                         & $\vdots$  & $\vdots$                   & \\
      \cline{1-2} \vline & m         & $\varphi$                  & & add. ass.  \\
      \vline             & $ \vdots$ & $\vdots$                   & \\
      \vline             & n-1       & $\psi$                     & & \\ \hline
                         & n         & $\varphi \rightarrow \psi$ & & $I_{\rightarrow}$
    \end{tabular}
  \end{minipage}
}
\bigskip

We can use this rule to show that $\vdash (p \wedge q) \rightarrow q$:

\begin{tabular}{cclcl}
   \cline{1-2} \vline & 1. & $p \wedge q$                        & & ass.  \\
   \vline             & 2. & $q$                                 & & $E_{\wedge}$, 1 \\ \hline
                      & 3. & $\vdash (p \wedge q) \rightarrow q$ & & $I_{\rightarrow}$ \\
\end{tabular}

Another example, with explicit assumptions given is the following derivation showing that  $(p \wedge q) \rightarrow r \vdash (q \wedge p) \rightarrow r$:

\begin{tabular}{cclcl}
                     & 1. & $(p \wedge q) \rightarrow r$ & & ass. \\
  \cline{1-2} \vline & 2. & $q \wedge p$                 & & ass. \\
  \vline             & 3. & $q$                          & & $E_{\wedge}$, 2  \\
  \vline             & 4. & $p$                          & & $E_{\wedge}$, 3  \\
  \vline             & 5. & $p \wedge q$                 & & $I_{\wedge}$, 4, 3  \\
  \vline             & 6. & $r$                          & & $E_{\rightarrow}$, 1, 5  \\ \hline
                     & 7. & $(q \wedge p) \rightarrow r$ & & $I_{\rightarrow}$
\end{tabular}



\newpage

\noindent \textbf{\underline{Introductieregel $I_{\rightarrow}$}}:



\noindent $I_{\rightarrow}$ mag alleen met de laatste assumptie worden gebruikt.
\bigskip
\bigskip
\bigskip

\noindent \textbf{\underline{Introductieregel $I_{\vee}$}}:

\fbox{\begin{tabular}{clcl}
  1. & . &  \\
   & . &  \\
   & . &  \\
  m & $\varphi$  &  \\
   & . & \\
   & . &  \\
   n. & $\varphi \vee \psi$ & & $I_{\vee}$, m
\end{tabular}}
\bigskip
\fbox{\begin{tabular}{clcl}
  1. & . &  \\
   & . &  \\
   & . &  \\
  m & $\varphi$  &  \\
   & . & \\
   & . &  \\
   n. & $\psi \vee \varphi$  & & $I_{\vee}$, m
\end{tabular}}
\newpage

\noindent \textbf{\underline{Eliminatieregel $E_{\vee}$}}:

\fbox{\begin{tabular}{clcl}
  1. & . &  \\
   & . &  \\
   & . &  \\
  m$_{1}$ & $\varphi \vee \psi$  &  \\
  & . &  \\
   & . &  \\
  m$_{2}$ & $\varphi \rightarrow \chi$  &  \\
   & . &  \\
   & . &  \\
  m$_{3}$ & $\psi  \rightarrow \chi$  &  \\
   & . & \\
   & . &  \\
   n. & $\chi$ & & $E_{\vee}$, m$_{1}$, m$_{2}$, m$_{3}$
\end{tabular}}
\bigskip

\noindent \textbf{\underline{Eliminatieregel $E_{\neg}$}}:

\fbox{\begin{tabular}{clcl}
  1. & . &  \\
   & . &  \\
   & . &  \\
  m$_{1}$ & $\neg \varphi$  &  \\
  & . &  \\
   & . &  \\
  m$_{2}$ & $\varphi$  &  \\
   & . & \\
   & . &  \\
   n. & $\bot$ & & $E_{\neg}$, m$_{1}$, m$_{2}$
\end{tabular}}
\bigskip

\noindent \textbf{\underline{Introductieregel $I_{\neg}$}}:

\fbox{\begin{tabular}{cclcl}
  & 1. & . &  \\
  & & . &  \\
  & & . &  \\
 \cline{1-2} \vline & m & $\varphi$  & & ass.  \\
 \vline & & . & &  \\
 \vline & & . &  &  \\
 \vline & n-1 & $\bot$ & & \\ \hline
 & n. & $\neg \varphi$ & & $I_{\neg}$
\end{tabular}}
\bigskip

\noindent \textbf{\underline{dubbelnegatie regel $\neg\neg$}}:

\fbox{
\begin{tabular}{clcl}
  1. & . &  \\
   & . &  \\
   & . &  \\
  m & $\neg\neg \varphi$  &  \\
   & . & \\
   & . &  \\
   n. & $\varphi$ & & $\neg\neg$, m
\end{tabular}}

\noindent \textbf{\underline{Introductieregel $I_{\exists}$}}:

\fbox{
\begin{tabular}{clcl}
  1. & . &  \\
   & . &  \\
   & . &  \\
  m & $[a/x] \varphi$  &  \\
   & . & \\
   & . &  \\
   n. & $\exists x \varphi$ & & $I_{\exists}$, m
\end{tabular}}

\noindent \textbf{\underline{Eliminatieregel $E_{\forall}$}}:

\fbox{
\begin{tabular}{clcl}
  1. & . &  \\
   & . &  \\
   & . &  \\
  m & $\forall x \varphi$  &  \\
   & . & \\
   & . &  \\
   n. & $[a/x] \varphi$ & & $E_{\forall}$, m
\end{tabular}}




\end{document}

