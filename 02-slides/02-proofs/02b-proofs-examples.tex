\documentclass[fleqn,10pt,serif,xcolor=svgnames,xcolor=table,aspectratio=169,handout]{beamer}
% \includeonlyframes{current}
%========================================
% Packages
%========================================

\usepackage[palatino]{../../99-auxiliary-files/00-mypackBeamer}
\usepackage{../../99-auxiliary-files/00-mycommands}
\usepackage{../../99-auxiliary-files/00-myenvironments-beamer}
%========================================
% More Layout (Beamer Special)
%========================================

\DefineNamedColor{named}{mycol}{cmyk}{0.6,0.6,0,0}
% \DefineNamedColor{named}{mygray}{cmyk}{0.05,0.05,0.05,0.05}
% \DefineNamedColor{named}{mygraylight}{cmyk}{0.017,0.017,0.017,0.017}

\definecolor{signal1}{rgb}{0.69, 0.25, 0.21}
\definecolor{signal2}{rgb}{1.0, 0.66, 0.07}
\definecolor{signal3}{rgb}{0.39, 0.58, 0.93}
\definecolor{signal4}{rgb}{0.0, 0.4, 0.0}
\definecolor{firebrick}{rgb}{0.7, 0.13, 0.13}
\definecolor{themecolor}{rgb}{0.3, 0.36, 0.33} % feldgrau
\definecolor{darkgray}{rgb}{0.66, 0.66, 0.66}

% \usetheme[height=7mm]{Rochester}
%\usetheme{Warsaw}


\usecolortheme{dove}

% \useoutertheme[compress,subsection=false]{miniframes}

\usecolortheme[named=themecolor]{structure}

\setbeamercolor{title}{fg=themecolor}

% \setbeamercolor{lower separation line head}{bg=white}

%\setbeamercolor{structure}{fg=Brown}
%\setbeamercolor{normal text}{fg=Brown}
%\setbeamercolor{section in head/foot}{bg=gray!40}
%%\setbeamercolor{lower separation line head}{bg=black!40}
%\setbeamercolor*{frametitle}{fg=Black,bg=gray!40}
%\setbeamercolor*{block body}{fg=Brown,bg=gray!00}
%\setbeamercolor*{block title}{fg=Black,bg=gray!40}


% Switch of shadows of boxes
\setbeamertemplate{blocks}[default]

% Frame numbers in footer
\setbeamertemplate{footline}[frame number]

% See-through preview for uncovered
% \setbeamercovered{transparent}

% Switch off navigation panel at bottom right
\beamertemplatenavigationsymbolsempty

% Change Style for itemize markers
% Options are ball, circle, rectangle and default (=triangle)
\setbeamertemplate{items}[circle]



\setcounter{tocdepth}{1}

% Use bullets in enumerates and TOC
\setbeamertemplate{enumerate item}[circle]

% Set color for enumerate/TOC bullets to white
\setbeamercolor*{item projected}{fg=themecolor,bg=gray!00}

\setbeamercolor*{author}{fg=gray!80}

\setbeamerfont*{block title}{size=\normalsize}
\setbeamerfont*{title}{size=\huge}
\setbeamerfont*{subtitle}{size=\large}

% \newcommand{\mygray}[1]{{\color{gray}{#1}}}
% \newcommand{\mycol}[1]{{\color{mycol}{#1}}}

\newcommand{\mycomment}[1]{\hfill {\mygray{#1}}}
\newcommand{\mycom}[1]{\hfill {\mygray{[#1]}}}

\newcommand{\slideFN}[1]{%
  \begin{textblock*}{\paperwidth}(0pt,1.05\textheight)
    \hfill \footnotesize{\mygray{#1}} \hspace{.5em}
  \end{textblock*}}

\newcommand{\pictureslide}[2][current]{
\usebackgroundtemplate{\includegraphics[width=\paperwidth]{#2}}%
\begin{frame}[label=#1]

\end{frame}
}
% code below makes it possible to turn inclusion of frames
% into 'miniframes' off and on with commands:
% \miniframeson and \miniframesoff
% from: http://tex.stackexchange.com/questions/37127/how-to-remove-some-pages-from-the-navigation-bullets-in-beamer

\makeatletter
\let\beamer@writeslidentry@miniframeson=\beamer@writeslidentry
\def\beamer@writeslidentry@miniframesoff{%
  \expandafter\beamer@ifempty\expandafter{\beamer@framestartpage}{}% does not happen normally
  {%else
    % removed \addtocontents commands
    \clearpage\beamer@notesactions%
  }
}
\newcommand*{\miniframeson}{\let\beamer@writeslidentry=\beamer@writeslidentry@miniframeson}
\newcommand*{\miniframesoff}{\let\beamer@writeslidentry=\beamer@writeslidentry@miniframesoff}
\makeatother

\setbeamertemplate{bibliography item}{}

\usepackage{array}
\usepackage[absolute,overlay]{textpos}

\usepackage{ulem}

% \usetheme[]{boxes}

%========================================
% Commands
%========================================

\newcommand{\mycol}[1]{{\textcolor{mycol}{#1}}}
\renewcommand{\markdef}[1]{\mycol{#1}}
\newcommand{\mygray}[1]{\textcolor{gray}{#1}}

\renewcommand{\slideFN}[1]{%
  \begin{textblock*}{\paperwidth}(0pt,0.95\textheight)
    \hfill \footnotesize{\mygray{#1}} \hspace{.5em}
  \end{textblock*}}

%========================================
% Document
%========================================

\title{Proof strategies: Examples}
\subtitle{Methods: Logic, Part 2b}

\author{Michael Franke}
\date{}


%--------------------------------------

\begin{document}

% --- Horizontal Space Fix ----

\abovedisplayskip=3pt
\abovedisplayshortskip=3pt

\belowdisplayskip=3pt
\belowdisplayshortskip=3pt

\begin{frame}
  \maketitle
\end{frame}

\begin{frame}
  \frametitle{Proof strategies}

  \begin{enumerate}[(i)]
    \item refutation by counterexample
    \item direct proof
    \item indirect proof
    \item inductive proof
  \end{enumerate}

\end{frame}

\begin{frame}
  \frametitle{Refutation by counterexample}

  \begin{mfproposition}
    \label{prop:refut-count-1}
    The following claim is false: \\
    For any sets $X$ and $Y$, if $X \in Y$, then all the elements of $X$ are also elements of $Y$.
  \end{mfproposition}

  \pause

  \begin{proof}
    A counterexample to the claim in question is given by the following two sets:
    \begin{enumerate}[]
      \item $X = \set{a,b}$
      \item $Y = \set{c, d, X} = \set{c, d, \set{a,b}}$
    \end{enumerate}
    Although $X \in Y$ and $a \in X$, it is not true that $a \in Y$.
  \end{proof}
\end{frame}

\begin{frame}
  \frametitle{Direct proof}

  \begin{mfproposition}
    \label{prop:emptyset-subset-of-any-X}
    For any set $X$, $\emptyset \subseteq X$.
  \end{mfproposition}

  \pause

  \begin{proof}
    Consider an arbitrary set $X$.\\ \pause
    For a set $Y$ be a subset of $X$, it is required that all elements of $Y$ are also in $X$.\\ \pause
    I.o.w., there cannot be a single element $y \in Y$ for which $y \not \in X$. \\ \pause
    Since the empty set contains no elements at all, there cannot be any element in it, which is not also in $X$.
  \end{proof}

\end{frame}

\begin{frame}
  \frametitle{Indirect proof}

  \begin{mfproposition}
    \label{prop:emptyset-subset-of-any-X}
    For any set $X$, $\emptyset \subseteq X$.
  \end{mfproposition}

  \pause

  \begin{proof}
    Assume that there is an $X$ for which $\es \not \subseteq X$. \\ \pause
    Then there must be an element in $\es$ which is not in $X$. \\ \pause
    But there are no elements in $\es$. So, we have a contradiction.
  \end{proof}

\end{frame}


\begin{frame}
  \frametitle{Inductive proof}

  \begin{minipage}{0.45\linewidth}
    \begin{mfdefinition}
      \begin{enumerate}
        \item \textbf{anchor:} the symbol ``*'' is part of $\mathcal{F}$
        \item \textbf{step:} if $f \in \mathcal{F}$, then so is ``(x)''
        \item \textbf{exhaustion:} nothing else is in $\mathcal{F}$
      \end{enumerate}
    \end{mfdefinition}
  \end{minipage}
  \hfill
  \pause
  \begin{minipage}{0.45\linewidth}
    \begin{mfproposition}
      Each $f \in \mathcal{F}$ has an equal number of opening and closing parentheses.
    \end{mfproposition}
  \end{minipage}

\pause

  \begin{proof}
    The inductive proof is over the number $n$ of opening parentheses.

    \emph{Inductive base.}
    If $f \in \mathcal{F}$ has no opening parenthesis, it must be $f = *$, for which the number of opening and closing parentheses is equal.\\ \pause

    \emph{Inductive assumption.} Any $f \in \mathcal{F}$ with $n = k-1$ opening parentheses has the same number of opening and closing parentheses. \\ \pause

    \emph{Inductive step.}
    If $f \in \mathcal{F}$ has $n=k$ opening parentheses, $f$ must be of the form $f = \text{``(g)''}$ where string $g \in \mathcal{F}$ has $k-1$ opening parentheses. \\ \pause
    By inductive assumption, $g$ has the same amount of opening and closing parentheses. \\ \pause
    But since $f = \text{``(g)''}$, and so exactly one parenthesis of each type is added to $g$, $f$ must have an equal number of parentheses, too.
  \end{proof}

\end{frame}
\end{document}
