\documentclass[fleqn,10pt,serif,xcolor=svgnames,xcolor=table,aspectratio=169,handout]{beamer}
% \includeonlyframes{current}
%========================================
% Packages
%========================================

\usepackage[palatino]{../../99-auxiliary-files/00-mypackBeamer}
\usepackage{../../99-auxiliary-files/00-mycommands}
\usepackage{../../99-auxiliary-files/00-myenvironments-beamer}

\usepackage{tikz-qtree}
\usepackage{array}
\usepackage[absolute,overlay]{textpos}
\usepackage{ulem}

\usepackage{pgfplots}

%========================================
% More Layout (Beamer Special)
%========================================

\DefineNamedColor{named}{mycol}{cmyk}{0.6,0.6,0,0}
% \DefineNamedColor{named}{mygray}{cmyk}{0.05,0.05,0.05,0.05}
% \DefineNamedColor{named}{mygraylight}{cmyk}{0.017,0.017,0.017,0.017}

\definecolor{signal1}{rgb}{0.69, 0.25, 0.21}
\definecolor{signal2}{rgb}{1.0, 0.66, 0.07}
\definecolor{signal3}{rgb}{0.39, 0.58, 0.93}
\definecolor{signal4}{rgb}{0.0, 0.4, 0.0}
\definecolor{firebrick}{rgb}{0.7, 0.13, 0.13}
\definecolor{themecolor}{rgb}{0.3, 0.36, 0.33} % feldgrau
\definecolor{darkgray}{rgb}{0.66, 0.66, 0.66}

% \usetheme[height=7mm]{Rochester}
%\usetheme{Warsaw}


\usecolortheme{dove}

% \useoutertheme[compress,subsection=false]{miniframes}

\usecolortheme[named=themecolor]{structure}

\setbeamercolor{title}{fg=themecolor}

% \setbeamercolor{lower separation line head}{bg=white}

%\setbeamercolor{structure}{fg=Brown}
%\setbeamercolor{normal text}{fg=Brown}
%\setbeamercolor{section in head/foot}{bg=gray!40}
%%\setbeamercolor{lower separation line head}{bg=black!40}
%\setbeamercolor*{frametitle}{fg=Black,bg=gray!40}
%\setbeamercolor*{block body}{fg=Brown,bg=gray!00}
%\setbeamercolor*{block title}{fg=Black,bg=gray!40}


% Switch of shadows of boxes
\setbeamertemplate{blocks}[default]

% Frame numbers in footer
\setbeamertemplate{footline}[frame number]

% See-through preview for uncovered
% \setbeamercovered{transparent}

% Switch off navigation panel at bottom right
\beamertemplatenavigationsymbolsempty

% Change Style for itemize markers
% Options are ball, circle, rectangle and default (=triangle)
\setbeamertemplate{items}[circle]



\setcounter{tocdepth}{1}

% Use bullets in enumerates and TOC
\setbeamertemplate{enumerate item}[circle]

% Set color for enumerate/TOC bullets to white
\setbeamercolor*{item projected}{fg=themecolor,bg=gray!00}

\setbeamercolor*{author}{fg=gray!80}

\setbeamerfont*{block title}{size=\normalsize}
\setbeamerfont*{title}{size=\huge}
\setbeamerfont*{subtitle}{size=\large}

% \newcommand{\mygray}[1]{{\color{gray}{#1}}}
% \newcommand{\mycol}[1]{{\color{mycol}{#1}}}

\newcommand{\mycomment}[1]{\hfill {\mygray{#1}}}
\newcommand{\mycom}[1]{\hfill {\mygray{[#1]}}}

\newcommand{\slideFN}[1]{%
  \begin{textblock*}{\paperwidth}(0pt,1.05\textheight)
    \hfill \footnotesize{\mygray{#1}} \hspace{.5em}
  \end{textblock*}}

\newcommand{\pictureslide}[2][current]{
\usebackgroundtemplate{\includegraphics[width=\paperwidth]{#2}}%
\begin{frame}[label=#1]

\end{frame}
}
% code below makes it possible to turn inclusion of frames
% into 'miniframes' off and on with commands:
% \miniframeson and \miniframesoff
% from: http://tex.stackexchange.com/questions/37127/how-to-remove-some-pages-from-the-navigation-bullets-in-beamer

\makeatletter
\let\beamer@writeslidentry@miniframeson=\beamer@writeslidentry
\def\beamer@writeslidentry@miniframesoff{%
  \expandafter\beamer@ifempty\expandafter{\beamer@framestartpage}{}% does not happen normally
  {%else
    % removed \addtocontents commands
    \clearpage\beamer@notesactions%
  }
}
\newcommand*{\miniframeson}{\let\beamer@writeslidentry=\beamer@writeslidentry@miniframeson}
\newcommand*{\miniframesoff}{\let\beamer@writeslidentry=\beamer@writeslidentry@miniframesoff}
\makeatother

\setbeamertemplate{bibliography item}{}


%========================================
% Commands
%========================================

\newcommand{\mycol}[1]{{\textcolor{mycol}{#1}}}
\renewcommand{\markdef}[1]{\textcolor{themecolor}{\textbf{#1}}}
\newcommand{\mygray}[1]{\textcolor{gray}{#1}}
\definecolor{darkgray}{rgb}{0.66, 0.66, 0.66}

\renewcommand{\slideFN}[1]{%
  \begin{textblock*}{\paperwidth}(0pt,0.95\textheight)
    \hfill \footnotesize{\mygray{#1}} \hspace{.5em}
  \end{textblock*}}

\newcommand{\proplog}{\acro{PropLog}}
\newcommand{\predlog}{\acro{PredLog}}

\newcommand{\mult}{\ensuremath{\cdot}}
\def\checkmark{\tikz\fill[scale=0.4](0,.35) -- (.25,0) -- (1,.7) -- (.25,.15) -- cycle;}

%========================================
% Document
%========================================

\title{Methods 1: Logic}
\subtitle{Introduction, overview, \& practicalities}

\author{Michael Franke}
\date{}


%--------------------------------------

\begin{document}

% --- Horizontal Space Fix ----

\abovedisplayskip=3pt
\abovedisplayshortskip=3pt

\belowdisplayskip=3pt
\belowdisplayshortskip=3pt

\begin{frame}
  \maketitle
\end{frame}


\begin{frame}
  \frametitle{Logic puzzle}

  There are two villages.
  In the honest village ($H$) everybody always speaks the truth.
  In the dishonest village ($D$) everybody always says the opposite of what is true.

  Before you the road splits: one way leads to the honest, the other to the dishonest village.
  At the splitting there is a man.
  He may be from village $H$ or $D$, you don't know.

  What do you ask the man to find out where the honest village is?

  \bigskip \pause

  \begin{center}
    \begin{tabular}{>{\columncolor{olive!15}}cc>{\columncolor{olive!15}}c}
      honest village & man       & where're you from? \\ \midrule
      left           & honest    & ``left'' \\
      left           & dishonest & ``left'' \\
      right          & honest    & ``right''\\
      right          & dishonest & ``right''
    \end{tabular}
  \end{center}

\end{frame}

\begin{frame}
  \frametitle{What is logic?}

  \begin{center}
    \begin{Large}
      \begin{tabular}{p{2cm}p{4cm}p{2cm}}
        proof & entailment & meaning
      \end{tabular}
    \end{Large}
  \end{center}

  \bigskip

  \begin{center}
    All Europeans are human.\\
    All humans are mortal\\
    Therefore, all Europeans are mortal.
  \end{center}
\end{frame}

\begin{frame}
  \frametitle{What is \textbf{a} logic?}

  \begin{itemize}
    \item there are different kinds of logic
    \item a logic is a formal system that captures some structural properties of meaning
    \item this course will cover three logics:
    \begin{enumerate}
      \item propositional logic \mycom{meaning of connectives \textit{and}, \textit{or}, \textit{not} \dots}
      \item predicate logic \mycom{meaning of quantifiers \textit{all}, \textit{some}, \textit{none} \dots}
      \item modal logic \mycom{meaning of epistemic attitudes \textit{belief}, \textit{knowledge} \dots}
    \end{enumerate}
  \end{itemize}

\end{frame}

\begin{frame}
  \frametitle{Course content}

  set theory

  3 logics

  probability

  information theory

\end{frame}

\begin{frame}
  \frametitle{Modeling}

insert model plane picture

\end{frame}

\begin{frame}
  \frametitle{Normative models}

\end{frame}

\begin{frame}
  \frametitle{Practicalities}
  \begin{itemize}
    \item enroll for this course on \textbf{moodle}:
    \begin{itemize}
      \item \url{https://moodle.zdv.uni-tuebingen.de/course/view.php?id=2876}
    \end{itemize}
    \item necessary for
    \begin{itemize}
      \item assessing course material
      \item receiving notifications
      \item asking questions in the forum
      \item submitting homework
      \item receiving feedback on homework
    \end{itemize}
  \end{itemize}
\end{frame}

\begin{frame}
  \frametitle{Best practice guide}
  \begin{enumerate}
    \item self-study
    \begin{itemize}
      \item prepare the assigned reading material \emph{before} the lecture
      \item bring questions, know what you don't know, ask and probe
    \end{itemize}
    \item lecture
    \begin{itemize}
      \item provides motivation, context and overview
      \item focuses on conceptual understanding
    \end{itemize}
    \item homework \hfill \mycom{start as early as possible each week}
    \begin{itemize}
      \item discussion with others is allowed \& encouraged
      \item write-up \& submissions must be made individually
      \item ask general questions on moodle, but do not share solutions
    \end{itemize}
    \item tutorials \hfill \mycom{go to at least one tutorial every week!}
    \begin{itemize}
      \item start working on homework questions \emph{before} the tutorial(s)
      \item emphasis on hands-on support for exercises
    \end{itemize}
  \end{enumerate}
\end{frame}

\begin{frame}
  \frametitle{Homework}
  \begin{itemize}
    \item \textbf{no copying from others} \hfill \mycom{plagiarism will lead to failure}
    \item submission: {\textcolor{red}{fill me}}
    \begin{itemize}
      \item where, when, how to submit?
    \end{itemize}
  \end{itemize}
\end{frame}

\begin{frame}
  \frametitle{Exam}
  \begin{itemize}
    \item February 16 2023, 10:00-16:00 (CET)
    \item take-home exam:
    \begin{itemize}
      \item released electronically at 10:00 via moodle
      \item solvable in ca.~3 hours
      \item you may use any material you like (books, handouts, \dots)
      \item cooperation is forbidden, submissions may not be copied
      \item submit electronically at 16:00 the latest via moodle
    \end{itemize}
  \end{itemize}
\end{frame}

\end{document}
